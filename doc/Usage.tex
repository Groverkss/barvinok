\section{\texorpdfstring{Usage of the \protect\ai[\tt]{barvinok} library}
{Usage of the barvinok library}}
\label{a:usage}

\index{barvinok@{\tt  barvinok}!availability}
{\sloppy 
This section describes some application programs
provided by the \barvinok/ library,
available from {\tt http://freshmeat.net/projects/barvinok/.}
For compilation instructions we refer to the \verb+README+ file
included in the distribution.
}

The program \ai[\tt]{barvinok\_count} enumerates a
non-parametric polytope.  It takes one polytope
in \PolyLib/ notation as input and prints the number
of integer points in the polytope and the time taken
by both ``manual counting'' and Barvinok's method.
The \PolyLib/ notation corresponds to the internal
representation of \ai[\tt]{Polyhedron}s as explained
in Appendix~\ref{a:existing}.
The first line of the input contains the number of rows
and the number of columns in the \ai[\tt]{Constraint} matrix.
The rest of the input is composed of the elements of the matrix.
Recall that the number of columns is two more than the number
of variables, where the extra first columns is one or zero
depending on whether the constraint is an inequality ($\ge 0$)
or an equality ($= 0$).  The next columns contain
the coefficients of the variables and the final column contains
the constant in the constraint.
E.g., the set 
$S = \lb\, s \mid s \geq 0 \wedge  2 s \leq 13 \, \rb$
from Example~\pref{ex:S:7} corresponds to the following input and
output.
\begin{verbatim}
> cat S
2 3

1 1 0
1 -2 13
> ./barvinok_count  < S
POLYHEDRON Dimension:1
           Constraints:2  Equations:0  Rays:2  Lines:0
Constraints 2 3
Inequality: [   1   0 ]
Inequality: [  -2  13 ]
Rays 2 3
Vertex: [   0 ]/1
Vertex: [  13 ]/2
manual:    7
User: 0.01; Sys: 0
Barvinok:    7
User: 0; Sys: 0
\end{verbatim}
The program \ai[\tt]{cdd2polylib.pl} can be used to
convert a polytope from \ai[\tt]{cdd} \shortcite{cdd}
notation to \PolyLib/ notation.

The program \ai[\tt]{barvinok\_enumerate} enumerates a
parametric polytope.  It takes two polytopes in \PolyLib/
notation as input, optionally followed by a list of parameter
names.
The two polytopes refer to arguments \verb+P+ and \verb+C+
of the corresponding function. (See Appendix~\ref{a:counting:functions}.)
The following example was taken by \shortciteN{Loechner1999}
from \shortciteN[Chapter II.2]{Loechner97phd}.
\begin{verbatim}
> cat loechner 
# Dimension of the matrix: 
7 7 
# Constraints: 
# i j k P Q cte 
1 1 0 0 0 0 0 # 0 <= i 
1 -1 0 0 1 0 0 # i <= P 
1 0 1 0 0 0 0 # 0 <= j 
1 1 -1 0 0 0 0 # j <= i
1 0 0 1 0 0 0 # 0 <= k 
1 1 -1 -1 0 0 0 # k <= i-j 
0 1 1 1 0 -1 0 # Q = i + j + k

# 2 parameters, no constraints. 
0 4
> ./barvinok_enumerate < loechner 
POLYHEDRON Dimension:5
           Constraints:6  Equations:1  Rays:5  Lines:0
Constraints 6 7
Equality:   [   1   1   1   0  -1   0 ]
Inequality: [   0   1   1   1  -1   0 ]
Inequality: [   0   1   0   0   0   0 ]
Inequality: [   0   0   1   0   0   0 ]
Inequality: [   0  -2  -2   0   1   0 ]
Inequality: [   0   0   0   0   0   1 ]
Rays 5 7
Ray:    [   1   0   1   1   2 ]
Ray:    [   1   1   0   1   2 ]
Vertex: [   0   0   0   0   0 ]/1
Ray:    [   0   0   0   1   0 ]
Ray:    [   1   0   0   1   1 ]
POLYHEDRON Dimension:2
           Constraints:1  Equations:0  Rays:3  Lines:2
Constraints 1 4
Inequality: [   0   0   1 ]
Rays 3 4
Line:   [   1   0 ]
Line:   [   0   1 ]
Vertex: [   0   0 ]/1
         - P + Q  >= 0
         2P - Q  >= 0
          1 >= 0

( -1/2 * P^2 + ( 1 * Q + 1/2 )
 * P + ( -3/8 * Q^2 + ( -1/2 * {( 1/2 * Q + 0 )
} + 1/4 )
 * Q + ( -5/4 * {( 1/2 * Q + 0 )
} + 1 )
 )
 )
         Q  >= 0
         P - Q  -1 >= 0
          1 >= 0

( 1/8 * Q^2 + ( -1/2 * {( 1/2 * Q + 0 )
} + 3/4 )
 * Q + ( -5/4 * {( 1/2 * Q + 0 )
} + 1 )
 )
\end{verbatim}
The output corresponds to 
$$
\begin{cases}
\makebox[0pt][l]{$-\frac 1 2 P^2 + P Q + \frac 1 2 P - \frac 3 8 Q^2
+ \left( \frac 1 4 - \frac 1 2 \left\{ \frac 1 2 Q \right\} \right)
	       	Q + 1 
- \frac 5 4 \left\{ \frac 1 2 Q \right\}$} \\
&
\hbox{if $P \le Q \le 2 P$}
\\
\frac 1 8 Q^2 + 
\left( \frac 3 4 - \frac 1 2 \left\{ \frac 1 2 Q \right\} \right)
- \frac 5 4 \left\{ \frac 1 2 Q \right\}
\qquad
\qquad
\qquad
\qquad
\qquad
&
\hbox{if $0 \le Q \le P-1$}
.
\end{cases}
$$

The program \ai[\tt]{barvinok\_enumerate\_e} enumerates a
parametric projected set.  It takes a single polytopes in \PolyLib/
notation as input, followed by two lines indicating the number
or existential variables and the number of parameters and
optionally followed by a list of parameter names.
The syntax for the line indicating the number of existential
variables is the letter \verb+E+ followed by a space and the actual number.
For indicating the number of parameters, the letter \verb+P+ is used.
The following example corresponds to Example~\pref{ex:S}.
\begin{verbatim}
> cat projected 
5 6
#   k   i   j   p   cst
1   0   1   0   0   -1
1   0   -1  0   0   8
1   0   0   1   0   -1
1   0   0   -1  1   0
0   -1  6   9   0   -7

E 2
P 1
> ./barvinok_enumerate_e <projected 
POLYHEDRON Dimension:4
           Constraints:5  Equations:1  Rays:4  Lines:0
Constraints 5 6
Equality:   [   1  -6  -9   0   7 ]
Inequality: [   0   1   0   0  -1 ]
Inequality: [   0  -1   0   0   8 ]
Inequality: [   0   0   1   0  -1 ]
Inequality: [   0   0  -1   1   0 ]
Rays 4 6
Vertex: [  50   8   1   1 ]/1
Ray:    [   0   0   0   1 ]
Ray:    [   9   0   1   1 ]
Vertex: [   8   1   1   1 ]/1
exist: 2, nparam: 1
         P  -3 >= 0
          1 >= 0

( 3 * P + 10 )
         P  -1 >= 0
         - P + 2 >= 0

( 8 * P + 0 )
\end{verbatim}

The program \ai[\tt]{barvinok\_series} enumerates a
parametric polytope in the form of a \rgf/.  
The input of this program is the same as that of
\ai[\tt]{barvinok\_enumerate}, except that the input polyhedron
is assumed to be full-dimensional.
The following is an example of Petr Lison\u{e}k\index{Lison\u{e}k, P.}.
\begin{verbatim}
> cat petr
4 6
1 -1 -1 -1 1 0
1 1 -1 0 0 0
1 0 1 -1 0 0
1 0 0 1 0 -1

0 3
n
> ./barvinok_series  < petr
POLYHEDRON Dimension:4
           Constraints:5  Equations:0  Rays:5  Lines:0
Constraints 5 6
Inequality: [  -1  -1  -1   1   0 ]
Inequality: [   1  -1   0   0   0 ]
Inequality: [   0   1  -1   0   0 ]
Inequality: [   0   0   1   0  -1 ]
Inequality: [   0   0   0   0   1 ]
Rays 5 6
Ray:    [   1   1   1   3 ]
Ray:    [   1   1   0   2 ]
Ray:    [   1   0   0   1 ]
Ray:    [   0   0   0   1 ]
Vertex: [   1   1   1   3 ]/1
POLYHEDRON Dimension:1
           Constraints:1  Equations:0  Rays:2  Lines:1
Constraints 1 3
Inequality: [   0   1 ]
Rays 2 3
Line:   [   1 ]
Vertex: [   0 ]/1
(n^3)/((1-n) * (1-n) * (1-n^2) * (1-n^3))
\end{verbatim}
