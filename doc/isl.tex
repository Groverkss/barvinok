\section{\protect\isl/ interface}

\let\llt\prec
\let\lle\preccurlyeq
\let\lgt\succ

\subsection{Library}

The \barvinok/ library currently supports only a few
functions that interface with the \isl/ library.
In time, this interface will grow and is set to replace
the \PolyLib/ interface.
For more information on the \isl/ data structures, see
the \isl/ user manual.

\begin{verbatim}
__isl_give isl_pw_qpolynomial *isl_set_card(__isl_take isl_set *set);
__isl_give isl_union_pw_qpolynomial *isl_union_set_card(
        __isl_take isl_union_set *uset);
\end{verbatim}
Compute the number of elements in an \ai[\tt]{isl\_set}
or \ai[\tt]{isl\_union\_set}.
The resulting \ai[\tt]{isl\_pw\_qpolynomial}
or \ai[\tt]{isl\_union\_pw\_qpolynomial} has purely parametric cells.

\begin{verbatim}
__isl_give isl_pw_qpolynomial *isl_map_card(__isl_take isl_map *map);
__isl_give isl_union_pw_qpolynomial *isl_union_map_card(
        __isl_take isl_union_map *umap);
\end{verbatim}
Compute a closed form expression for the number of image elements
associated to any element in the domain of the given \ai[\tt]{isl\_map}
or \ai[\tt]{isl\_union\_map}.
The union of the cells in the resulting \ai[\tt]{isl\_pw\_qpolynomial}
is equal to the domain of the input \ai[\tt]{isl\_map}.

\begin{verbatim}
__isl_give isl_pw_qpolynomial *isl_pw_qpolynomial_sum(
        __isl_take isl_pw_qpolynomial *pwqp);
__isl_give isl_union_pw_qpolynomial *isl_union_pw_qpolynomial_sum(
        __isl_take isl_union_pw_qpolynomial *upwqp);
\end{verbatim}
Compute the sum of the given piecewise quasipolynomial over
all integer points in the domain.  The result is a piecewise
quasipolynomial that only involves the parameters.
If, however, the domain of the piecewise quasipolynomial wraps
a relation, then the sum is computed over all integer points
in the range of that relation and the domain of the relation
becomes the domain of the result.

\begin{verbatim}
__isl_give isl_pw_qpolynomial *isl_map_apply_pw_qpolynomial(
        __isl_take isl_map *map, __isl_take isl_pw_qpolynomial *pwqp);
__isl_give isl_union_pw_qpolynomial *isl_union_map_apply_union_pw_qpolynomial(
        __isl_take isl_union_map *umap,
        __isl_take isl_union_pw_qpolynomial *upwqp);
\end{verbatim}
Compose the given map with the given piecewise quasipolynomial.
That is, compute the sum over all elements in the intersection
of the range of the map and the domain of the piecewise quasipolynomial
as a function of an element in the domain of the map.

\subsection{Calculator}

The \ai[\tt]{iscc} calculator offers an interface to some
of the functionality provided by the \isl/, \cloog/ and \barvinok/
libraries.
The language used by \ai[\tt]{iscc} is extremely simple.  The calculator
supports operations on constants and dynamically typed variables and
assignments (\ai[\tt]{:=}) to those variables.  If the result of an expression
is not used inside another expression and not assigned to a variable,
then this result is printed on the screen.  The operators are overloaded
based on the types of the arguments, which may be sets, relations,
piecewise quasipolynomials, piecewise quasipolynomial folds, lists,
strings or booleans.
The supported operations are shown in \autoref{t:iscc}.
Note that when an operation requires an argument of a certain
type, a binary list with the first element of the required type
may also be used instead.

\subsubsection{Objects and Basic Operations}

A set contains integer tuples that satisfy some affine constraints
in disjunctive normal form (DNF).  The description may involve (free)
parameters and existentially quantified variables.
For example, the set
\begin{verbatim}
[n] -> {[i] : exists a : i = 2a and 0 <= i <= n; D[i,j] : i,j >= 0}
\end{verbatim}
contains all the even integer numbers between 0 and the value of
the parameter $n$, as well
as the positive orthant of a 2-dimensional \verb!D! space.
The optional parameters should
be declared by placing them in a comma delimited list inside \verb![]!
(followed by an ``\verb!->!'') in front of the main set description.
The parameters are global and are identified by their names,
so the order inside the list is arbitrary.
The main set description consists of a \verb!{}!-pair containing
a semicolon delimited list of
subsets from a given space.  A space is identified by its optional
name (\verb!D! in the example) and its dimension (the number
of coordinates in the \verb![]! enclosed tuple following the optional name).
The names of the coordinates are purely local within the set description.
The space specification is separated by a colon from the actual constraints in DNF
describing the elements from that space.
The disjuncts are delimited by \verb!or!, while the conjuncts are delimited
by \verb!and!.  Each disjunct may be preceded by \verb!exists!
followed by a list of existentially quantified variables and a colon.

Maps are binary relations on pairs of spaces and are defined in a similar
way, except that the single space is replaced by a pair of spaces separated
by \verb!->!.  Maps need not be single-valued.
Each space may contain a nested pair of spaces.  Such nested spaces
are extremely useful in more advanced applications.
As an example, suppose that during equivalence checking
\shortcite{Verdoolaege2009equivalence}
of two programs the iterations of \verb!S1! in one program are supposed to
produce the same results as the same iterations of \verb!SA! in the other program,
which may be described using the following map
\begin{verbatim}
[n] -> { S1[i] -> SA[i] : 0 <= i <= n }
\end{verbatim}
If the iterations of \verb!S1! depend on the same iterations
of \verb!S2!, i.e., \verb!{S1[i]->S2[i]}!, while those of \verb!SA!
depend on the next iteration of \verb!B!, i.e., \verb!{SA[i]->SB[i+1]}!,
then we can apply the cross product of these two dependence maps, i.e.,
\begin{verbatim}
{ [S1[i] -> SA[i']] -> [S2[i] -> SB[1+i']] }
\end{verbatim}
to the original map to find
out which iterations of \verb!S2! should correspond to which
iterations of \verb!SB!.

Basic operations on sets and maps include intersection (\ai[\tt]{*}),
union (\ai[\tt]{+}), difference (\ai[\tt]{-}), cross product (\ai[\tt]{cross}),
sampling (\ai[\tt]{sample}), affine hull (\ai[\tt]{aff}),
lexicographic optimization (\ai[\tt]{lexmin} or \ai[\tt]{lexmax}),
subset (\ai[\tt]{<=}) and equality (\ai[\tt]{=}) tests,
code generation (\ai[\tt]{codegen})
and the cardinality (\ai[\tt]{card}).
Additional operations on maps include computing domain (\ai[\tt]{dom})
and range (\ai[\tt]{ran}), differences between image and domain (\ai[\tt]{deltas}),
join (\ai[\tt]{.}), inverse (\ai[\tt]{\^{}-1}) and transitive closure (\ai[\tt]{\^{}+}).
The latter may result in an overapproximation.

The \ai[\tt]{card} operation computes the number of elements in a set
or the number of elements in the image of a domain element of a map.
The operation is performed exactly and completely symbolically and
the result is a piecewise quasipolynomial, i.e., a subdivision of one
or more spaces, with a quasipolynomial associated to each cell in the subdivision.
As a trivial example, the result of
\begin{verbatim}
card { A[i] -> B[j] : 0 <= j <= i }
\end{verbatim}
is \verb!{ A[i] -> (1+i) : i >= 0 }!.
Operations on piecewise quasipolynomials include sum (\ai[\tt]{+})
and difference (\ai[\tt]{-}) and the computation of an upper bound over the domain.
If the domain contains a pair of nested spaces, then the upper bound is computed over
the nested range.  As another trivial example, the result of
\begin{verbatim}
ub{ [[i] -> [j]] -> j^2 : -i <= j <= i }
\end{verbatim}
is
\verb!({ [i] -> max(i^2) : i >= 0 }, True)!.
The first element in this list is the actual bound in the form
of a piecewise quasipolynomial fold,
i.e., a maximum of quasipolynomials defined over cells.
The second indicates whether the bound is tight, i.e., whether
a maximum has been computed.

\subsubsection{Advanced Operations}

While the basic {\tt card} operation simply counts the number of elements
in an affine set, it is also possible to assign a weight to each element
of the set and to compute the sum of those weights over all the points in the set.
The syntax for this weighted counting is to compute the {\tt sum} of
a piecewise quasipolynomial over its domain.  As in the case of the {\tt ub}
operator, if the domain contains a pair of nested space, the sum is computed
over the range.  As an example, the result
of
\begin{verbatim}
sum{ [[i] -> [j]] -> i*j : 0 <= j <= i }
\end{verbatim}
is
\verb|{ [i] -> (1/2*i^2+1/2*i^3) : i >= 0 }|.

After the computation of some sum or bound, the result may have to
be reformulated in terms of other variables.  For example, during
inter procedural analysis, a result computed in terms of the formal
parameters may have to be reformulated in terms of the actual parameters.
{\tt iscc} therefore allows maps and
piecewise quasipolynomials or folds to be composed.
If the map is multi-valued, then the composition maps each domain element
to the sum or upper bound of the values at its image elements.

Finally, because of its high-level representation, {\tt iscc} can
provide a dependence analysis operation taking only three maps as input,
the sink accesses, the potential source accesses and a schedule.
The result is a single dependence map.


\subsubsection{More Examples}
\begin{verbatim}
P := [n, m] -> { [i,j] : 0 <= i <= n and i <= j <= m };
card P;

f := [n,m] -> { [i,j] -> i*j + n*i*i*j : i,j >= 0 and 5i + 27j <= n+m };
sum f;
s := sum f;
s @ [n,m] -> { [] : 0 <= n,m <= 20 };

f := [n] -> { [i] -> 2*n*i - n*n + 3*n - 1/2*i*i - 3/2*i-1 :
                (exists j : 0 <= i < 4*n-1 and 0 <= j < n and
                            2*n-1 <= i+j <= 4*n-2 and i <= 2*n-1 ) };
ub f;
u := (ub f)[0];
u @ [n] -> { [] : 0 <= n <= 10 };

m := [n] -> { [i,j] -> [i+1,j+1] : 1 <= i,j < n;
              [i,j] -> [i+1,j-1] : 1 <= i < n and 2 <= j <= n };
m^+;
(m^+)[0];

codegen [N] -> { A[i] -> [i,0] : 0 <= i <= N; B[i] -> [i,1] : 1 <= i <= N };

{ [k] -> [i] : 1 <= i <= k } . { [n] -> 2 * n : n >= 1 };

{ [m] -> [c] : 1 <= c <= m } . { [k] -> max((3 * k + k^2)) : k >= 1 };
\end{verbatim}

\subsubsection{Comparison to other Interactive Polyhedral Tools}

Two related interactive polyhedral tools are
the Omega calculator \shortcite{Omega_calc}
and {\tt SPPoC} \shortcite{Boulet2001SPPoC}.
The syntax of {\tt iscc} was very much inspired
by that of the Omega calculator.  However, the Omega
calculator only knows sets and maps.  In particular,
it does not perform any form of counting.  An earlier version
of \barvinok/ came with a modified version of
the Omega calculator that introduced an operation
for counting the number of elements in a set, but it
would simply print the result and not allow any further
manipulations.
{\tt SPPoC} does support counting, but only the basic
operation of counting the elements in a set.
In particular, it does not support weighted counting,
nor the computation of upper bounds.
It also only supports (single-valued) functions
and not generic relations like the Omega calculator and {\tt iscc}.
Internally, {\tt SPPoC} uses {\tt PolyLib}, {\tt PipLib} and {\tt omega}
to perform
its operations.  Although the first two of these libraries may have been
state-of-the-art at the time {\tt SPPoC} was created, they are
no longer actively maintained and have been largely
superseded by more recent libraries.
In particular, {\tt PipLib} effectively only supports a single
operation, which is now also available in both {\tt isl} and {\tt PPL}.
The operations on rational polyhedra in {\tt PolyLib} are also
available in {\tt PPL}, usually through a cleaner interface and
with a more efficient implementation.  As to counting the elements
in a parametric polytope, Barvinok's algorithm,
implemented in the {\tt barvinok} library, is usually much faster
than the algorithm implemented in {\tt PolyLib}
\shortcite{Verdoolaege2007parametric}.
Furthermore,
the ability to work with named and nested spaces and the ability
of sets and maps to contain (pairs of) elements from different spaces
are not available in the Omega calculator and {\tt SPPoC}.

\newpage
\tablecaption{{\tt iscc} operations.  The variables
have the following types,
$s$: set,
$m$: map,
$q$: piecewise quasipolynomial,
$f$: piecewise quasipolynomial fold,
$l$: list,
$i$: integer,
$b$: boolean,
$S$: string,
$o$: object of any type
}
\label{t:iscc}
\tablehead{
\hline
Syntax & Meaning
\\
\hline
}
\tabletail{
\multicolumn{2}{r}{\small\sl continued on next page}
\\
}
\tablelasttail{}
\begin{supertabular}{p{0.35\textwidth}p{0.6\textwidth}}
$s_2$ := \ai[\tt]{aff} $s_1$ & affine hull of $s_1$
\\
$m_2$ := \ai[\tt]{aff} $m_1$ & affine hull of $m_1$
\\
$q$ := \ai[\tt]{card} $s$ &
number of elements in the set $s$
\\
$q$ := \ai[\tt]{card} $m$ &
number of elements in the image of a domain element
\\
$s_2$ := \ai[\tt]{coalesce} $s_1$ &
simplify the representation of set $s_1$ by trying
to combine pairs of basic sets into a single
basic set
\\
$m_2$ := \ai[\tt]{coalesce} $m_1$ &
simplify the representation of map $m_1$ by trying
to combine pairs of basic maps into a single
basic map
\\
$q_2$ := \ai[\tt]{coalesce} $q_1$ &
simplify the representation of $q_1$ by trying
to combine pairs of basic sets in the domain
of $q_1$ into a single basic set
\\
$f_2$ := \ai[\tt]{coalesce} $f_1$ &
simplify the representation of $f_1$ by trying
to combine pairs of basic sets in the domain
of $f_1$ into a single basic set
\\
\ai[\tt]{codegen} $s$ &
generate code for the given domain.
This operation is only available if \ai[\tt]{CLooG}
support was compiled in.
\\
\ai[\tt]{codegen} $m$ &
generate code for the given scattering function.
This operation is only available if \ai[\tt]{CLooG}
support was compiled in.
\\
$s_3$ := $s_1$ \ai[\tt]{cross} $s_2$ &
Cartesian product of $s_1$ and $s_2$
\\
$m_3$ := $m_1$ \ai[\tt]{cross} $m_2$ &
Cartesian product of $m_1$ and $m_2$
\\
$s$ := \ai[\tt]{deltas} $m$ &
the set $\{\, y - x \mid x \to y \in m \,\}$
\\
$s$ := \ai[\tt]{dom} $m$ &
domain of map $m$
\\
$s$ := \ai[\tt]{dom} $q$ &
domain of piecewise quasipolynomial $q$
\\
$s$ := \ai[\tt]{dom} $f$ &
domain of piecewise quasipolynomial fold $f$
\\
$s$ := \ai[\tt]{ran} $m$ &
range of map $m$
\\
$m$ := \ai[\tt]{identity} $s$ &
identity relation on $s$
\\
$s_2$ := \ai[\tt]{lexmin} $s_1$ &
lexicographically minimal element of $s_1$
\\
$m_2$ := \ai[\tt]{lexmin} $m_1$ &
lexicographically minimal image element
\\
$s_2$ := \ai[\tt]{lexmax} $s_1$ &
lexicographically maximal element of $s_1$
\\
$m_2$ := \ai[\tt]{lexmax} $m_1$ &
lexicographically maximal image element
\\
$s_2$ := \ai[\tt]{poly} $s_1$ & polyhedral hull of $s_1$
\\
$m_2$ := \ai[\tt]{poly} $m_1$ & polyhedral hull of $m_1$
\\
$o$ := \ai[\tt]{read} {\tt "}{\it filename}{\tt"} &
read object from file
\\
$s_2$ := \ai[\tt]{sample} $s_1$ &
a sample element of the set $s_1$
\\
$m_2$ := \ai[\tt]{sample} $m_1$ &
a sample element of the map $m_1$
\\
$s_2$ := \ai[\tt]{scan} $s_1$ &
the set $s_1$ split into individual elements,
provided $s_1$ contains a finite number of elements
\\
$m_2$ := \ai[\tt]{scan} $m_1$ &
the map $m_1$ split into individual elements,
provided $m_1$ contains a finite number of elements
\\
\ai[\tt]{source} {\tt "}{\it filename}{\tt"} &
read commands from file
\\
$q_2$ := \ai[\tt]{sum} $q_1$ &
sum $q_1$ over all integer points in the domain of $q_1$,
or, if the domain of $q_1$ wraps a map, over all integer
points in the range of the wrapped map.
\\
$l$ := \ai[\tt]{ub} $q$ &
compute an
upper bound on the piecewise quasipolynomial $q$ over
all integer points in the domain of $q$
and return a list containing the upper bound
and a boolean that is true if the upper bound
is known to be tight.
If the domain of $q$ wraps a map, then the upper
bound is computed over all integer points in
the range of the wrapped map instead.
\\
$l$ := \ai[\tt]{vertices} $s$ &
a list of vertices of the rational polytope defined by the same constraints
as $s$
\\
$s$ := \ai[\tt]{wrap} $m$ &
wrap the map in a set
\\
$m$ := \ai[\tt]{unwrap} $s$ &
unwrap the map from the set
\\
$m_4$ := \ai[\tt]{any} $m_1$ \ai[\tt]{before} $m_2$ \ai[\tt]{under} $m_3$ &
compute a map from any element $x$ in the domain of $m_1$
to any element $y$ in the domain of $m_2$
such that their images $m_1(x)$ and $m_2(y)$ overlap
and such that $m_3(x) \llt m_3(y)$.
\\
$l$ := \ai[\tt]{last} $m_1$ \ai[\tt]{before} $m_2$ \ai[\tt]{under} $m_3$ &
compute a map that contains for any element $y$ in the domain of $m_2$
a mapping from the lexicographically last element $x$ in the domain of $m_1$
(according to the schedule $m_3$) to $y$
such that $m_1(x)$ and $m_2(y)$ overlap and such that $m_3(x) \llt m_3(y)$.
Return a list containing this map and the set of elements in the domain of
$m_2$ for which there is no corresponding element in the domain of $m_1$.
\\
$m_5$ := \ai[\tt]{any} $m_1$ \ai[\tt]{last} $m_2$ \ai[\tt]{before} $m_3$
\ai[\tt]{under} $m_4$ &
compute a map that contains for any element $y$ in the domain of $m_3$
a mapping from the lexicographically last element $x$ in the domain of $m_2$
(according to the schedule $m_4$) to $y$
such that $m_2(x)$ and $m_3(y)$ overlap and such that $m_4(x) \llt m_4(y)$
as well as from any element $z$ in the domain of $m_1$ such that
$m_1(z)$ and $m_3(y)$ overlap, $m_4(z) \llt m_4(y)$ and such that there
is no $x$ in the domain of $m_2$ with
$m_2(x) \cap m_3(y) \ne \emptyset$ and $m_4(z) \llt m_4(x) \llt m_4(y)$.
\\
$s_3$ := $s_1$ \ai{$+$} $s_2$ & union
\\
$m_3$ := $m_1$ \ai{$+$} $m_2$ & union
\\
$q_3$ := $q_1$ \ai{$+$} $q_2$ & sum
\\
$f_2$ := $f_1$ \ai{$+$} $q$ & sum
\\
$f_2$ := $q$ \ai{$+$} $f_1$ & sum
\\
$S_3$ := $S_1$ \ai{$+$} $S_2$ & concatenation
\\
$S_2$ := $o$ \ai{$+$} $S_1$ &
concatenation of stringification of $o$ and $S_1$
\\
$S_2$ := $S_1$ \ai{$+$} $o$ &
concatenation of $S_1$ and stringification of $o$
\\
$s_3$ := $s_1$ \ai{$-$} $s_2$ & set difference
\\
$m_3$ := $m_1$ \ai{$-$} $m_2$ & set difference
\\
$q_3$ := $q_1$ \ai{$-$} $q_2$ & difference
\\
$s_3$ := $s_1$ \ai{$*$} $s_2$ & intersection
\\
$m_3$ := $m_1$ \ai{$*$} $m_2$ & intersection
\\
$q_3$ := $q_1$ \ai{$*$} $q_2$ & product
\\
$m_2$ := $m_1$ \ai{$*$} $s$ & intersect domain of $m_1$ with $s$
\\
$q_2$ := $q_1$ \ai{$*$} $s$ & intersect domain of $q_1$ with $s$
\\
$f_2$ := $f_1$ \ai{$*$} $s$ & intersect domain of $f_1$ with $s$
\\
$s_2$ := $m$($s_1$) & apply map $m$ to set $s_1$
\\
$m_3$ := $m_1$ \ai[\tt]{.} $m_2$ & join of $m_1$ and $m_2$
\\
$m_3$ := $m_1$ \ai[\tt]{before} $m_2$ & join of $m_1$ and $m_2$
\\
$m_3$ := $m_2$($m_1)$ & join of $m_1$ and $m_2$
\\
$m_3$ := $m_2$ \ai[\tt]{after} $m_1$ & join of $m_1$ and $m_2$
\\
$f_3$ := $f_1$ \ai[\tt]{.} $f_2$ & join of $f_1$ and $f_2$, combining
the lists of quasipolynomials over shared domains
\\
$q_2$ := $m$ \ai[\tt]{.} $q_1$ & join of $m$ and $q_1$, taking the sum
over all elements in the intersection of the range of $m$ and the domain
of $q_1$
\\
$q_2$ := $m$ \ai[\tt]{before} $q_1$ & $q_2$ := $m$ \ai[\tt]{.} $q_1$
\\
$q_2$ := $q_1(m)$ & $q_2$ := $m$ \ai[\tt]{.} $q_1$
\\
$q_2$ := $q_1$ \ai[\tt]{after} $m$ & $q_2$ := $m$ \ai[\tt]{.} $q_1$
\\
$f_2$ := $m$ \ai[\tt]{.} $f_1$ & join of $m$ and $f_1$, computing a bound
over all elements in the intersection of the range of $m$ and the domain
of $f_1$
\\
$f_2$ := $m$ \ai[\tt]{before} $f_1$ & $f_2$ := $m$ \ai[\tt]{.} $f_1$
\\
$f_2$ := $f_1(m)$ & $f_2$ := $m$ \ai[\tt]{.} $f_1$
\\
$f_2$ := $f_1$ \ai[\tt]{after} $m$ & $f_2$ := $m$ \ai[\tt]{.} $f_1$
\\
$m$ := $s_1$ \ai[\tt]{->} $s_2$ & universal map with domain $s_1$
and range $s_2$
\\
$q_2$ := $q_1$ \ai{@} $s$ &
evaluate the piecewise quasipolynomial $q_1$ in each element
of the set $s$ and return a piecewise quasipolynomial
mapping each of the individual elements to the resulting
constant
\\
$q$ := $f$ \ai{@} $s$ &
evaluate the piecewise quasipolynomial fold $f$ in each element
of the set $s$ and return a piecewise quasipolynomial
mapping each of the individual elements to the resulting
constant
\\
$s_3$ := $s_1$ \ai[\tt]{\%} $s_2$ &
simplify $s_1$ in the context of $s_2$, i.e., compute
the gist of $s_1$ given $s_2$
\\
$m_3$ := $m_1$ \ai[\tt]{\%} $m_2$ &
simplify $m_1$ in the context of $m_2$, i.e., compute
the gist of $m_1$ given $m_2$
\\
$q_2$ := $q_1$ \ai[\tt]{\%} $s$ &
simplify $q_1$ in the context of the domain $s$, i.e., compute
the gist of $q_1$ given $s$
\\
$f_2$ := $f_1$ \ai[\tt]{\%} $s$ &
simplify $f_1$ in the context of the domain $s$, i.e., compute
the gist of $f_1$ given $s$
\\
$m_2$ := $m_1$\ai[\tt]{\^{}-1} & inverse of $m_1$
\\
$l$ := $m$\ai[\tt]{\^{}+} &
compute an overapproximation of the transitive closure
of $m$ and return a list containing the overapproximation
and a boolean that is true if the overapproximation
is known to be exact
\\
$o$ := $l$[$i$] &
the element at position $i$ in the list $l$
\\
$b$ := $s_1$ \ai[\tt]{=} $s_2$ & is $s_1$ equal to $s_2$?
\\
$b$ := $m_1$ \ai[\tt]{=} $m_2$ & is $m_1$ equal to $m_2$?
\\
$b$ := $s_1$ \ai[\tt]{<=} $s_2$ & is $s_1$ a subset of $s_2$?
\\
$b$ := $m_1$ \ai[\tt]{<=} $m_2$ & is $m_1$ a subset of $m_2$?
\\
$b$ := $s_1$ \ai[\tt]{<} $s_2$ & is $s_1$ a proper subset of $s_2$?
\\
$b$ := $m_1$ \ai[\tt]{<} $m_2$ & is $m_1$ a proper subset of $m_2$?
\\
$b$ := $s_1$ \ai[\tt]{>=} $s_2$ & is $s_1$ a superset of $s_2$?
\\
$b$ := $m_1$ \ai[\tt]{>=} $m_2$ & is $m_1$ a superset of $m_2$?
\\
$b$ := $s_1$ \ai[\tt]{>} $s_2$ & is $s_1$ a proper superset of $s_2$?
\\
$b$ := $m_1$ \ai[\tt]{>} $m_2$ & is $m_1$ a proper superset of $m_2$?
\\
$m$ := $s_1$ \ai[\tt]{<<} $s_2$ & a map from
$s_1$ to $s_2$ of those elements that live in the same space and
such that the elements of $s_1$ are lexicographically strictly smaller
than those of $s_2$.
\\
$m_3$ := $m_1$ \ai[\tt]{<<} $m_2$ & a map from the domain of
$m_1$ to the domain of $m_2$ of those elements such that their images
live in the same space and such that the images of the elements of
$m_1$ are lexicographically strictly smaller than those of $m_2$.
\\
$m$ := $s_1$ \ai[\tt]{<<=} $s_2$ & a map from
$s_1$ to $s_2$ of those elements that live in the same space and
such that the elements of $s_1$ are lexicographically smaller
than those of $s_2$.
\\
$m_3$ := $m_1$ \ai[\tt]{<<=} $m_2$ & a map from the domain of
$m_1$ to the domain of $m_2$ of those elements such that their images
live in the same space and such that the images of the elements of
$m_1$ are lexicographically smaller than those of $m_2$.
\\
$m$ := $s_1$ \ai[\tt]{>>} $s_2$ & a map from
$s_1$ to $s_2$ of those elements that live in the same space and
such that the elements of $s_1$ are lexicographically strictly greater
than those of $s_2$.
\\
$m_3$ := $m_1$ \ai[\tt]{>>} $m_2$ & a map from the domain of
$m_1$ to the domain of $m_2$ of those elements such that their images
live in the same space and such that the images of the elements of
$m_1$ are lexicographically strictly greater than those of $m_2$.
\\
$m$ := $s_1$ \ai[\tt]{>>=} $s_2$ & a map from
$s_1$ to $s_2$ of those elements that live in the same space and
such that the elements of $s_1$ are lexicographically greater
than those of $s_2$.
\\
$m_3$ := $m_1$ \ai[\tt]{>>=} $m_2$ & a map from the domain of
$m_1$ to the domain of $m_2$ of those elements such that their images
live in the same space and such that the images of the elements of
$m_1$ are lexicographically greater than those of $m_2$.
\\
\end{supertabular}
