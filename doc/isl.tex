\section{\protect\isl/ interface}

\subsection{Library}

The \barvinok/ library currently supports just two
functions that interface with the \isl/ library.
In time, this interface will grow and is set to replace
the \PolyLib/ interface.
For more information on the \isl/ data structures, see
the \isl/ user manual.

\begin{verbatim}
__isl_give isl_pw_qpolynomial *isl_set_card(__isl_take isl_set *set);
\end{verbatim}
Compute the number of elements in an \ai[\tt]{isl\_set}.
The resulting \ai[\tt]{isl\_pw\_qpolynomial} has purely parametric cells.

\begin{verbatim}
__isl_give isl_pw_qpolynomial *isl_map_card(__isl_take isl_map *map);
\end{verbatim}
Compute a closed form expression for the number of image elements
associated to any element in the domain of the given \ai[\tt]{isl\_map}.
The union of the cells in the resulting \ai[\tt]{isl\_pw\_qpolynomial}
is equal to the domain of the input \ai[\tt]{isl\_map}.

\begin{verbatim}
__isl_give isl_pw_qpolynomial *isl_pw_qpolynomial_sum(
        __isl_take isl_pw_qpolynomial *pwqp);
\end{verbatim}
Compute the sum of the given piecewise quasipolynomial over
all integer points in the domain.  The result is a piecewise
quasipolynomial that only involves the parameters.

\subsection{Calculator}

The \ai[\tt]{iscc} calculator offers an interface to some
of the functionality provided by the \isl/ and \barvinok/
libraries.
The supported operations are shown in \autoref{t:iscc}.
Here are some examples:
\begin{verbatim}
P := [n, m] -> { [i,j] : 0 <= i <= n and i <= j <= m };
card P;

f := [n,m] -> { [i,j] -> i*j + n*i*i*j : i,j >= 0 and 5i + 27j <= n+m };
sum f;
s := sum f;
s @ [n,m] -> { [] : 0 <= n,m <= 20 };

f := [n] -> { [i] -> 2*n*i - n*n + 3*n - 1/2*i*i - 3/2*i-1 :
                (exists j : 0 <= i < 4*n-1 and 0 <= j < n and
                            2*n-1 <= i+j <= 4*n-2 and i <= 2*n-1 ) };
ub f;
u := (ub f)[0];
u @ [n] -> { [] : 0 <= n <= 10 };

m := [n] -> { [i,j] -> [i+1,j+1] : 1 <= i,j < n;
              [i,j] -> [i+1,j-1] : 1 <= i < n and 2 <= j <= n };
m^+;
(m^+)[0];
\end{verbatim}

\bottomcaption{{\tt iscc} operations.  The variables
have the following types,
$s$: set,
$m$: map,
$q$: piecewise quasipolynomial,
$f$: piecewise quasipolynomial fold,
$l$: list,
$i$: integer,
$b$: boolean,
$o$: object of any type
}
\label{t:iscc}
\tablehead{
Syntax & Meaning
\\
\hline
}
\tabletail{
\multicolumn{2}{r}{\small\sl continued on next page}
\\
}
\tablelasttail{}
\begin{supertabular}{lp{0.7\textwidth}}
$s_2$ := \ai[\tt]{aff} $s_1$ & affine hull of $s_1$
\\
$m_2$ := \ai[\tt]{aff} $m_1$ & affine hull of $m_1$
\\
$q$ := \ai[\tt]{card} $s$ &
number of elements in the set $s$
\\
$q$ := \ai[\tt]{card} $m$ &
number of elements in the image of a domain element
\\
$s_2$ := \ai[\tt]{coalesce} $s_1$ &
simplify the representation of set $s_1$ by trying
to combine pairs of basic sets into a single
basic set
\\
$m_2$ := \ai[\tt]{coalesce} $m_1$ &
simplify the representation of map $m_1$ by trying
to combine pairs of basic maps into a single
basic map
\\
$q_2$ := \ai[\tt]{coalesce} $q_1$ &
simplify the representation of $q_1$ by trying
to combine pairs of basic sets in the domain
of $q_1$ into a single basic set
\\
$f_2$ := \ai[\tt]{coalesce} $f_1$ &
simplify the representation of $f_1$ by trying
to combine pairs of basic sets in the domain
of $f_1$ into a single basic set
\\
$s_3$ := $s_1$ \ai[\tt]{cross} $s_2$ &
Cartesian product of $s_1$ and $s_2$
\\
$m_3$ := $m_1$ \ai[\tt]{cross} $m_2$ &
Cartesian product of $m_1$ and $m_2$
\\
$s$ := \ai[\tt]{deltas} $m$ &
the set $\{\, y - x \mid x \to y \in m \,\}$
\\
$s$ := \ai[\tt]{dom} $m$ &
domain of map $m$
\\
$s$ := \ai[\tt]{dom} $q$ &
domain of piecewise quasipolynomial $q$
\\
$s$ := \ai[\tt]{dom} $f$ &
domain of piecewise quasipolynomial fold $f$
\\
$s$ := \ai[\tt]{ran} $m$ &
range of map $m$
\\
$s_2$ := \ai[\tt]{lexmin} $s_1$ &
lexicographically minimal element of $s_1$
\\
$m_2$ := \ai[\tt]{lexmin} $m_1$ &
lexicographically minimal image element
\\
$s_2$ := \ai[\tt]{lexmax} $s_1$ &
lexicographically maximal element of $s_1$
\\
$m_2$ := \ai[\tt]{lexmax} $m_1$ &
lexicographically maximal image element
\\
$o$ := \ai[\tt]{read} {\tt "}{\it filename}{\tt"} &
read object from file
\\
$s_2$ := \ai[\tt]{sample} $s_1$ &
a sample element of the set $s_1$
\\
$m_2$ := \ai[\tt]{sample} $m_1$ &
a sample element of the map $m_1$
\\
$q_2$ := \ai[\tt]{sum} $q_1$ &
sum $q_1$ over all integer points in the domain of $q_1$
\\
$l$ := \ai[\tt]{ub} $q$ &
compute an
upper bound on the piecewise quasipolynomial $q$ over
all integer points in the domain of $q$
and return a list containing the upper bound
and a boolean that is true if the upper bound
is known to be tight
\\
$l$ := \ai[\tt]{vertices} $s$ &
a list of vertices of the rational polytope defined by the same constraints
as $s$
\\
$s_3$ := $s_1$ \ai{$+$} $s_2$ & union
\\
$m_3$ := $m_1$ \ai{$+$} $m_2$ & union
\\
$q_3$ := $q_1$ \ai{$+$} $q_2$ & sum
\\
$s_3$ := $s_1$ \ai{$-$} $s_2$ & set difference
\\
$m_3$ := $m_1$ \ai{$-$} $m_2$ & set difference
\\
$q_3$ := $q_1$ \ai{$-$} $q_2$ & difference
\\
$s_3$ := $s_1$ \ai{$*$} $s_2$ & intersection
\\
$m_3$ := $m_1$ \ai{$*$} $m_2$ & intersection
\\
$q_3$ := $q_1$ \ai{$*$} $q_2$ & product
\\
$m_2$ := $m_1$ \ai{$*$} $s$ & intersect domain of $m_1$ with $s$
\\
$q_2$ := $q_1$ \ai{$*$} $s$ & intersect domain of $q_1$ with $s$
\\
$f_2$ := $f_1$ \ai{$*$} $s$ & intersect domain of $f_1$ with $s$
\\
$s_2$ := $m$($s_1$) & apply map $m$ to set $s_1$
\\
$m_3$ := $m_1$ \ai[\tt]{.} $m_2$ & join of $m_1$ and $m_2$
\\
$m_3$ := $m_2$($m_1)$ & join of $m_1$ and $m_2$
\\
$m$ := $s_1$ \ai[\tt]{->} $s_2$ & universal map with domain $s_1$
and range $s_2$
\\
$q_2$ := $q_1$ \ai{@} $s$ &
evaluate the piecewise quasipolynomial $q_1$ in each element
of the set $s$ and return a piecewise quasipolynomial
mapping each of the individual elements to the resulting
constant
\\
$q$ := $f$ \ai{@} $s$ &
evaluate the piecewise quasipolynomial fold $f$ in each element
of the set $s$ and return a piecewise quasipolynomial
mapping each of the individual elements to the resulting
constant
\\
$s_3$ := $s_1$ \ai[\tt]{\%} $s_2$ &
simplify $s_1$ in the context of $s_2$, i.e., compute
the gist of $s_1$ given $s_2$
\\
$m_3$ := $m_1$ \ai[\tt]{\%} $m_2$ &
simplify $m_1$ in the context of $m_2$, i.e., compute
the gist of $m_1$ given $m_2$
\\
$q_2$ := $q_1$ \ai[\tt]{\%} $s$ &
simplify $q_1$ in the context of the domain $s$, i.e., compute
the gist of $q_1$ given $s$
\\
$f_2$ := $f_1$ \ai[\tt]{\%} $s$ &
simplify $f_1$ in the context of the domain $s$, i.e., compute
the gist of $f_1$ given $s$
\\
$m_2$ := $m_1$\ai[\tt]{\^{}-1} & inverse of $m_1$
\\
$l$ := $m$\ai[\tt]{\^{}+} &
compute an overapproximation of the transitive closure
of $m$ and return a list containing the overapproximation
and a boolean that is true if the overapproximation
is known to be exact
\\
$o$ := $l$[$i$] &
the element at position $i$ in the list $l$
\\
$b$ := $s_1$ \ai[\tt]{=} $s_2$ & is $s_1$ equal to $s_2$?
\\
$b$ := $m_1$ \ai[\tt]{=} $m_2$ & is $m_1$ equal to $m_2$?
\\
$b$ := $s_1$ \ai[\tt]{<=} $s_2$ & is $s_1$ a subset of $s_2$?
\\
$b$ := $m_1$ \ai[\tt]{<=} $m_2$ & is $m_1$ a subset of $m_2$?
\\
$b$ := $s_1$ \ai[\tt]{<} $s_2$ & is $s_1$ a proper subset of $s_2$?
\\
$b$ := $m_1$ \ai[\tt]{<} $m_2$ & is $m_1$ a proper subset of $m_2$?
\\
$b$ := $s_1$ \ai[\tt]{>=} $s_2$ & is $s_1$ a superset of $s_2$?
\\
$b$ := $m_1$ \ai[\tt]{>=} $m_2$ & is $m_1$ a superset of $m_2$?
\\
$b$ := $s_1$ \ai[\tt]{>} $s_2$ & is $s_1$ a proper superset of $s_2$?
\\
$b$ := $m_1$ \ai[\tt]{>} $m_2$ & is $m_1$ a proper superset of $m_2$?
\\
\end{supertabular}
