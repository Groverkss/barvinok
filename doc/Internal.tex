\section{\protect\PolyLib/ interface of the \protect\ai[\tt]{barvinok} library
(obsolescent)}

Although \barvinok/ currently still uses \PolyLib/ internally,
this is likely to change in the not too distant future.
Consider using \isl/ based alternatives for the functions in this section
as the latter are likely to be removed in future releases.

Our \barvinok/ library is built on top of \PolyLib/ 
\shortcite{Wilde1993,Loechner1999}.
In particular, it reuses the implementations
of the algorithm of 
\shortciteN{Loechner97parameterized}
for computing parametric vertices
and the algorithm of
\shortciteN{Clauss1998parametric}
for computing chamber decompositions.
Initially, our library was meant to be a replacement
for the algorithm of \shortciteN{Clauss1998parametric},
also implemented in \PolyLib/, for computing quasi-polynomials.
To ease the transition of application programs we
tried to reuse the existing data structures as much as possible.

\subsection{Existing Data Structures}
\label{a:existing}

Inside \PolyLib/ integer values are represented by the 
\ai[\tt]{Value} data type.
Depending on a configure option, the data type may
either by a 32-bit integer, a 64-bit integer
or an arbitrary precision integer using \ai[\tt]{GMP}.
The \barvinok/ library requires that \PolyLib/ is compiled
with support for arbitrary precision integers.

The basic structure for representing (unions of) polyhedra is a
\ai[\tt]{Polyhedron}.
\begin{verbatim}
typedef struct polyhedron {
  unsigned Dimension, NbConstraints, NbRays, NbEq, NbBid;
  Value **Constraint;
  Value **Ray;
  Value *p_Init;
  int p_Init_size;
  struct polyhedron *next;
} Polyhedron;
\end{verbatim}
The attribute \ai[\tt]{Dimension} is the dimension
of the ambient space, i.e., the number of variables.
The attributes \ai[\tt]{Constraint}
and \ai[\tt]{Ray} point to two-dimensional arrays
of constraints and generators, respectively.
The number of rows is stored in
\ai[\tt]{NbConstraints} and
\ai[\tt]{NbRays}, respectively.
The number of columns in both arrays is equal
to \verb!1+Dimension+1!.
The first column of \ai[\tt]{Constraint} is either
$0$ or $1$ depending on whether the constraint 
is an equality ($0$) or an inequality ($1$).
The number of equalities is stored in \ai[\tt]{NbEq}.
If the constraint is $\sp a x + c \ge 0$, then
the next columns contain the coefficients $a_i$
and the final column contains the constant $c$.
The first column of \ai[\tt]{Ray} is either
$0$ or $1$ depending on whether the generator 
is a line ($0$) or a vertex or ray ($1$).
The number of lines is stored in \ai[\tt]{NbBid}.
Let $d$ be the \ac{lcm} of the denominators of the coordinates
of a vertex $\vec v$, then the next columns contain
$d v_i$ and the final column contains $d$.
For a ray, the final column contains $0$.
The field \ai[\tt]{next} points to the next polyhedron in
the union of polyhedra.
It is \verb+0+ if this is the last (or only) polyhedron in the union.
For more information on this structure, we refer to \shortciteN{Wilde1993}.

Quasi-polynomials are represented using the 
\ai[\tt]{evalue} and \ai[\tt]{enode} structures.
\begin{verbatim}
typedef enum { polynomial, periodic, evector } enode_type;

typedef struct _evalue {
  Value d;              /* denominator */
  union {
    Value n;            /* numerator (if denominator != 0) */
    struct _enode *p;   /* pointer   (if denominator == 0) */
  } x;
} evalue;

typedef struct _enode {
  enode_type type;      /* polynomial or periodic or evector */
  int size;             /* number of attached pointers */
  int pos;              /* parameter position */
  evalue arr[1];        /* array of rational/pointer */
} enode;
\end{verbatim}
If the field \ai[\tt]{d} of an \ai[\tt]{evalue} is zero, then
the \ai[\tt]{evalue} is a placeholder for a pointer to
an \ai[\tt]{enode}, stored in \ai[\tt]{x.p}.
Otherwise, the \ai[\tt]{evalue} is a rational number with
numerator \ai[\tt]{x.n} and denominator \ai[\tt]{d}.
An \ai[\tt]{enode} is either a \ai[\tt]{polynomial}
or a \ai[\tt]{periodic}, depending on the value
of \ai[\tt]{type}.
The length of the array \ai[\tt]{arr} is stored in \ai[\tt]{size}.
For a \ai[\tt]{polynomial}, \ai[\tt]{arr} contains the coefficients.
For a \ai[\tt]{periodic}, it contains the values for the different
residue classes modulo the parameter indicated by \ai[\tt]{pos}.
For a polynomial, \ai[\tt]{pos} refers to the variable
of the polynomial.
The value of \ai[\tt]{pos} is \verb+1+ for the first parameter.
That is, if the value of \ai[\tt]{pos} is \verb+1+ and the first
parameter is $p$, and if the length of the array is $l$,
then in case it is a polynomial, the
 \ai[\tt]{enode} represents
$$
\verb+arr[0]+ + \verb+arr[1]+ p + \verb+arr[2]+ p^2 + \cdots +
\verb+arr[l-1]+ p^{l-1}
.
$$
If it is a periodic, then it represents
$$
\left[
\verb+arr[0]+, \verb+arr[1]+,  \verb+arr[2]+, \ldots,
\verb+arr[l-1]+
\right]_p
.
$$
Note that the elements of a \ai[\tt]{periodic} may themselves
be other \ai[\tt]{periodic}s or even \ai[\tt]{polynomial}s.
In our library, we only allow the elements of a  \ai[\tt]{periodic}
to be other \ai[\tt]{periodic}s or rational numbers.
The chambers and their corresponding quasi-polynomial are
stored in \ai[\tt]{Enumeration} structures.
\begin{verbatim}
typedef struct _enumeration {
  Polyhedron *ValidityDomain; /* constraints on the parameters */
  evalue EP;                  /* dimension = combined space    */
  struct _enumeration *next;  /* Ehrhart Polynomial,
                                 corresponding to parameter
                                 values inside the domain
                                 ValidityDomain above          */
} Enumeration;
\end{verbatim}
For more information on these structures, we refer to \shortciteN{Loechner1999}.

\begin{example}
Figure~\ref{f:Loechner} is a skillful reconstruction
of Figure~2 from \shortciteN{Loechner1999}.
It shows the contents of the \ai[\tt]{enode} structures
representing the quasi-polynomial
$
[1,2]_p p^2 + 3 p + \frac 5 2
$.

\begin{figure}
\begin{xy}
\POS(0,0)*!UL{\hbox{
\tt
\begin{tabular}{|c|c|c|}
\hline
\multicolumn{2}{|c|}{type} & polynomial \\
\hline
\multicolumn{2}{|c|}{size} & 3 \\
\hline
\multicolumn{2}{|c|}{pos} & 1 \\
\hline
\smash{\lower 6.25pt\hbox{arr[0]}} & d & 2 \\
\cline{2-3}
       & x.n & 5 \\
\hline
\smash{\lower 6.25pt\hbox{arr[1]}} & d & 1 \\
\cline{2-3}
       & x.n & 3 \\
\hline
\smash{\lower 6.25pt\hbox{arr[2]}} & d & 0 \\
\cline{2-3}
       & x.p &   \\
\hline
\end{tabular}
}
}="box1"
+DR*!DR\hbox{\strut\hskip 1.5\tabcolsep\phantom{\tt polynomial}\hskip 1.5\tabcolsep}+C="a"
\POS(60,-15)*!UL{\hbox{
\tt
\begin{tabular}{|c|c|c|}
\hline
\multicolumn{2}{|c|}{type} & periodic \\
\hline
\multicolumn{2}{|c|}{size} & 2 \\
\hline
\multicolumn{2}{|c|}{pos} & 1 \\
\hline
\smash{\lower 6.25pt\hbox{arr[0]}} & d & 1 \\
\cline{2-3}
       & x.n & 1 \\
\hline
\smash{\lower 6.25pt\hbox{arr[1]}} & d & 1 \\
\cline{2-3}
       & x.n & 2 \\
\hline
\end{tabular}
}
}="box2"
+UL+<0.5\tabcolsep,0pt>*!UL\hbox{\strut}+CL="b"
\POS"a"\ar@(r,l) "b"
\POS"box1"+UC*++!D\hbox{\tt enode}
\POS"box2"+UC*++!D\hbox{\tt enode}
\end{xy}
\caption{The quasi-polynomial $[1,2]_p p^2 + 3 p + \frac 5 2$.}
\label{f:Loechner}
\end{figure}
\end{example}

\subsection{Options}
\label{a:options}

The \ai[\tt]{barvinok\_options} structure contains various
options that influence the behavior of the library.

\begin{verbatim}
struct barvinok_options {
    struct barvinok_stats   *stats;

    /* PolyLib options */
    unsigned    MaxRays;

    /* NTL options */
                /* LLL reduction parameter delta=LLL_a/LLL_b */
    long        LLL_a;
    long        LLL_b;

    /* barvinok options */
    #define	BV_SPECIALIZATION_BF		2
    #define	BV_SPECIALIZATION_DF		1
    #define	BV_SPECIALIZATION_RANDOM	0
    #define	BV_SPECIALIZATION_TODD		3
    int         incremental_specialization;

    unsigned long           max_index;
    int                     primal;
    int                     lookup_table;
    int                     count_sample_infinite;

    int                     try_Delaunay_triangulation;

    #define     BV_APPROX_SIGN_NONE     0
    #define     BV_APPROX_SIGN_APPROX   1
    #define     BV_APPROX_SIGN_LOWER    2
    #define     BV_APPROX_SIGN_UPPER    3
    int                     polynomial_approximation;
    #define     BV_APPROX_NONE          0
    #define     BV_APPROX_DROP          1
    #define     BV_APPROX_SCALE         2
    #define     BV_APPROX_VOLUME        3
    #define	BV_APPROX_BERNOULLI	4
    int                     approximation_method;
    #define     BV_APPROX_SCALE_FAST    (1 << 0)
    #define     BV_APPROX_SCALE_NARROW  (1 << 1)
    #define     BV_APPROX_SCALE_NARROW2 (1 << 2)
    #define     BV_APPROX_SCALE_CHAMBER (1 << 3)
    int                     scale_flags;
    #define     BV_VOL_LIFT             0
    #define     BV_VOL_VERTEX           1
    #define     BV_VOL_BARYCENTER       2
    int                     volume_triangulate;

    /* basis reduction options */
    #define     BV_GBR_NONE     0
    #define     BV_GBR_GLPK     1
    #define     BV_GBR_CDD      2
    int         gbr_lp_solver;

    #define     BV_LP_POLYLIB           0
    #define     BV_LP_GLPK              1
    #define     BV_LP_CDD               2
    #define     BV_LP_CDDF              3
    #define     BV_LP_PIP               4
    int         lp_solver;

    #define     BV_HULL_GBR             0
    #define     BV_HULL_HILBERT         1
    int         integer_hull;
};

struct barvinok_options *barvinok_options_new_with_defaults();
\end{verbatim}

The function \ai[\tt]{barvinok\_options\_new\_with\_defaults}
can be used to create a \ai[\tt]{barvinok\_options} structure
with default values.

\begin{itemize}
\item \PolyLib/ options

\begin{itemize}

\item \ai[\tt]{MaxRays}

The value of \ai[\tt]{MaxRays} is passed to various \PolyLib/
functions and defines the
maximum size of a table used in the \ai{double description} computation
in the \PolyLib/ function \ai[\tt]{Chernikova}.
In earlier versions of \PolyLib/,
this parameter had to be conservatively set
to a high number to ensure successful operation,
resulting in significant memory overhead.
Our change to allow this table to grow
dynamically is available in recent versions of \PolyLib/.
In these versions, the value no longer indicates the maximal
table size, but rather the size of the initial allocation.
This value may be set to \verb+0+ or left as set
by \ai[\tt]{barvinok\_options\_new\_with\_defaults}.

\end{itemize}

\item \ai[\tt]{NTL} options

\begin{itemize}

\item \ai[\tt]{LLL\_a}
\item \ai[\tt]{LLL\_b}

The values used for the \ai{reduction parameter}
in the call to \ai[\tt]{NTL}'s implementation of \indac{LLL}.

\end{itemize}

\item \ai[\tt]{barvinok} specific options

\begin{itemize}

\item \ai[\tt]{incremental\_specialization}

Selects the \ai{specialization} algorithm to be used.
If set to {\tt 0} then a direct specialization is performed
using a random vector.
Value {\tt 1} selects a depth first incremental specialization,
while value {\tt 2} selects a breadth first incremental specialization.
The default is selected by the \ai[\tt]{--enable-incremental}
\ai[\tt]{configure} option.
For more information we refer to~\citeN[Section~4.4.3]{Verdoolaege2005PhD}.

\end{itemize}

\end{itemize}

\subsection{Data Structures for Quasi-polynomials}
\label{a:data}

Internally, we do not represent our \ai{quasi-polynomial}s
as step-polynomials, but instead as polynomials of
fractional parts of degree-$1$ polynomials.
However, we also allow our quasi-polynomials to be represented
as polynomials with periodic numbers for coefficients,
similarly to \shortciteN{Loechner1999}.
By default, the current version of \barvinok/ uses
\ai[\tt]{fractional}s, but this can be changed through
the \ai[\tt]{--disable-fractional} configure option.
When this option is specified, the periodic numbers
are represented as
an explicit enumeration using the \ai[\tt]{periodic} type.
A quasi-polynomial based on fractional
parts can also be converted to an actual step-polynomial
using \ai[\tt]{evalue\_frac2floor}, but this is not fully
supported yet.

For reasons of compatibility,%
\footnote{Also known as laziness.}
we shoehorned our representations for piecewise quasi-polynomials
into the existing data structures.
To this effect, we introduced four new types,
\ai[\tt]{fractional}, \ai[\tt]{relation},
\ai[\tt]{partition} and \ai[\tt]{flooring}.
\begin{verbatim}
typedef enum { polynomial, periodic, evector, fractional,
               relation, partition, flooring } enode_type;
\end{verbatim}
The field \ai[\tt]{pos} is not used in most of these
additional types and is therefore set to \verb+-1+.

The types \ai[\tt]{fractional} and \ai[\tt]{flooring}
represent polynomial expressions in a fractional part or a floor respectively.
The generator is stored in \verb+arr[0]+, while the
coefficients are stored in the remaining array elements.
That is, an \ai[\tt]{enode} of type \ai[\tt]{fractional}
represents
$$
\verb+arr[1]+ + \verb+arr[2]+ \{\verb+arr[0]+\} + 
\verb+arr[3]+ \{\verb+arr[0]+\}^2 + \cdots +
\verb+arr[l-1]+ \{\verb+arr[0]+\}^{l-2}
.
$$
An \ai[\tt]{enode} of type \ai[\tt]{flooring}
represents
$$
\verb+arr[1]+ + \verb+arr[2]+ \lfloor\verb+arr[0]+\rfloor + 
\verb+arr[3]+ \lfloor\verb+arr[0]+\rfloor^2 + \cdots +
\verb+arr[l-1]+ \lfloor\verb+arr[0]+\rfloor^{l-2}
.
$$

\begin{example}
The internal representation of the quasi-polynomial
$$\left(1+2 \left\{\frac p 2\right\}\right) p^2 + 3 p + \frac 5 2$$
is shown in Figure~\ref{f:fractional}.

\begin{figure}
\begin{xy}
\POS(0,0)*!UL{\hbox{
\tt
\begin{tabular}{|c|c|c|}
\hline
\multicolumn{2}{|c|}{type} & polynomial \\
\hline
\multicolumn{2}{|c|}{size} & 3 \\
\hline
\multicolumn{2}{|c|}{pos} & 1 \\
\hline
\smash{\lower 6.25pt\hbox{arr[0]}} & d & 2 \\
\cline{2-3}
       & x.n & 5 \\
\hline
\smash{\lower 6.25pt\hbox{arr[1]}} & d & 1 \\
\cline{2-3}
       & x.n & 3 \\
\hline
\smash{\lower 6.25pt\hbox{arr[2]}} & d & 0 \\
\cline{2-3}
       & x.p &   \\
\hline
\end{tabular}
}
}="box1"
+DR*!DR\hbox{\strut\hskip 1.5\tabcolsep\phantom{\tt polynomial}\hskip 1.5\tabcolsep}+C="a"
\POS(60,0)*!UL{\hbox{
\tt
\begin{tabular}{|c|c|c|}
\hline
\multicolumn{2}{|c|}{type} & fractional \\
\hline
\multicolumn{2}{|c|}{size} & 3 \\
\hline
\multicolumn{2}{|c|}{pos} & -1 \\
\hline
\smash{\lower 6.25pt\hbox{arr[0]}} & d & 0 \\
\cline{2-3}
       & x.p &   \\
\hline
\smash{\lower 6.25pt\hbox{arr[1]}} & d & 1 \\
\cline{2-3}
       & x.n & 1 \\
\hline
\smash{\lower 6.25pt\hbox{arr[2]}} & d & 1 \\
\cline{2-3}
       & x.n & 2 \\
\hline
\end{tabular}
}
}="box2"
+UL+<0.5\tabcolsep,0pt>*!UL\hbox{\strut}+CL="b"
\POS"a"\ar@(r,l) "b"
\POS"box2"+UR*!UR{\hbox{
\tt
\begin{tabular}{|c|}
\hline
 fractional \\
\hline
 3 \\
\hline
 -1 \\
\hline
 0 \\
\hline
\end{tabular}
}
}+CD*!U{\strut}+C="c"
\POS(60,-50)*!UL{\hbox{
\tt
\begin{tabular}{|c|c|c|}
\hline
\multicolumn{2}{|c|}{type} & polynomial \\
\hline
\multicolumn{2}{|c|}{size} & 2 \\
\hline
\multicolumn{2}{|c|}{pos} & 1 \\
\hline
\smash{\lower 6.25pt\hbox{arr[0]}} & d & 1 \\
\cline{2-3}
       & x.n & 0 \\
\hline
\smash{\lower 6.25pt\hbox{arr[1]}} & d & 2 \\
\cline{2-3}
       & x.n & 1 \\
\hline
\end{tabular}
}
}="box3"
+UR-<0.8\tabcolsep,0pt>*!UR\hbox{\strut}+CR="d"
\POS"c"\ar@(r,r) "d"
\POS"box1"+UC*++!D\hbox{\tt enode}
\POS"box2"+UC*++!D\hbox{\tt enode}
\POS"box3"+UC*++!D\hbox{\tt enode}
\end{xy}
\caption{The quasi-polynomial 
$\left(1+2 \left\{\frac p 2\right\}\right) p^2 + 3 p + \frac 5 2$.}
\label{f:fractional}
\end{figure}

\end{example}

The \ai[\tt]{relation} type is used to represent \ai{stride}s.
In particular, if the value of \ai[\tt]{size} is 2, then
the value of a \ai[\tt]{relation} is (in pseudo-code):
\begin{verbatim}
(value(arr[0]) == 0) ? value(arr[1]) : 0
\end{verbatim}
If the size is 3, then the value is:
\begin{verbatim}
(value(arr[0]) == 0) ? value(arr[1]) : value(arr[2])
\end{verbatim}
The type of \verb+arr[0]+ is typically \ai[\tt]{fractional}.

Finally, the \ai[\tt]{partition} type is used to
represent piecewise quasi-polynomials.
We prefer to encode this information inside \ai[\tt]{evalue}s
themselves
rather than using \ai[\tt]{Enumeration}s since we want
to perform the same kinds of operations on both quasi-polynomials
and piecewise quasi-polynomials.
An \ai[\tt]{enode} of type \ai[\tt]{partition} may not be nested
inside another \ai[\tt]{enode}.
The size of the array is twice the number of ``chambers''.
Pointers to chambers are stored in the even slots,
whereas pointer to the associated quasi-polynomials
are stored in the odd slots.
To be able to store pointers to chambers, the
definition of \ai[\tt]{evalue} was changed as follows.
\begin{verbatim}
typedef struct _evalue {
  Value d;              /* denominator */
  union {
    Value n;            /* numerator (if denominator > 0) */
    struct _enode *p;   /* pointer   (if denominator == 0) */
    Polyhedron *D;      /* domain    (if denominator == -1) */
  } x;
} evalue;
\end{verbatim}
Note that we allow a ``chamber'' to be a union of polyhedra
as discussed in \citeN[Section~4.5.1]{Verdoolaege2005PhD}.
Chambers with extra variables, i.e., those of
\citeN[Section~4.6.5]{Verdoolaege2005PhD},
are only partially supported.
The field \ai[\tt]{pos} is set to the actual dimension,
i.e., the number of parameters.

\subsection{Operations on Quasi-polynomials}
\label{a:operations}

In this section we discuss some of the more important
operations on \ai[\tt]{evalue}s provided by the
\barvinok/ library.
Some of these operations are extensions
of the functions from \PolyLib/ with the same name.

Most of these operation are also provided by \isl/ on
\ai[\tt]{isl\_pw\_qpolynomial}s, which are set to replace
\ai[\tt]{evalue}s.  Use \ai[\tt]{isl\_pw\_qpolynomial\_from\_evalue} to convert
from \ai[\tt]{evalue}s to \ai[\tt]{isl\_pw\_qpolynomial}s.
\begin{verbatim}
__isl_give isl_pw_qpolynomial *isl_pw_qpolynomial_from_evalue(
        __isl_take isl_dim *dim, const evalue *e);
\end{verbatim}

\begin{verbatim}
void eadd(const evalue *e1,evalue *res);
void emul(const evalue *e1, evalue *res);
\end{verbatim}
The functions \ai[\tt]{eadd} and \ai[\tt]{emul} takes
two (pointers to) \ai[\tt]{evalue}s \verb+e1+ and \verb+res+
and computes their sum and product respectively.
The result is stored in \verb+res+, overwriting (and deallocating)
the original value of \verb+res+.
It is an error if exactly one of
the arguments of \ai[\tt]{eadd} is of type \ai[\tt]{partition}
(unless the other argument is \verb+0+).
The addition and multiplication operations are described
in \citeN[Section~4.5.1]{Verdoolaege2005PhD} 
and~\citeN[Section~4.5.2]{Verdoolaege2005PhD}
respectively.

The function \ai[\tt]{eadd} is an extension of the function
\ai[\tt]{new\_eadd} from \shortciteN{Seghir2002}.
Apart from supporting the additional types from Section~\ref{a:data},
the new version also additionally imposes an order on the nesting of
different \ai[\tt]{enode}s.
Without such an ordering, \ai[\tt]{evalue}s could be constructed
representing for example
$$
(0 y^ 0 + ( 0 x^0 + 1 x^1 ) y^1 ) x^0 + (0 y^0 - 1 y^1) x^1
,
$$
which is just a funny way of saying $0$.

\begin{verbatim}
void eor(evalue *e1, evalue *res);
\end{verbatim}
The function \ai[\tt]{eor} implements the \ai{union}
operation from \citeN[Section~4.5.3]{Verdoolaege2005PhD}.  Both arguments
are assumed to correspond to indicator functions.

\begin{verbatim}
evalue *esum(evalue *E, int nvar);
evalue *evalue_sum(evalue *E, int nvar, unsigned MaxRays);
\end{verbatim}
The function \ai[\tt]{esum} has been superseded by
\ai[\tt]{evalue\_sum}.
The function \ai[\tt]{evalue\_sum} performs the summation
operation from \citeN[Section~4.5.4]{Verdoolaege2005PhD}.
The piecewise step-polynomial represented by \verb+E+ is summated
over its first \verb+nvar+ variables.
Note that \verb+E+ must be zero or of type \ai[\tt]{partition}.
The function returns the result in a newly allocated 
\ai[\tt]{evalue}.
Note also that \verb+E+ needs to have been converted
from \ai[\tt]{fractional}s to \ai[\tt]{flooring}s using
the function \ai[\tt]{evalue\_frac2floor}.
\begin{verbatim}
void evalue_frac2floor(evalue *e);
\end{verbatim}
This function also ensures that the arguments of the
\ai[\tt]{flooring}s are positive in the relevant chambers.
It currently assumes that the argument of each
\ai[\tt]{fractional} in the original \ai[\tt]{evalue}
has a minimum in the corresponding chamber.

\begin{verbatim}
double compute_evalue(const evalue *e, Value *list_args);
Value *compute_poly(Enumeration *en,Value *list_args);
evalue *evalue_eval(const evalue *e, Value *values);
\end{verbatim}
The functions \ai[\tt]{compute\_evalue},
\ai[\tt]{compute\_poly} and
\ai[\tt]{evalue\_eval}
evaluate a (piecewise) quasi-polynomial
at a certain point.  The argument \verb+list_args+
points to an array of \ai[\tt]{Value}s that is assumed
to be long enough.
The \verb+double+ return value of \ai[\tt]{compute\_evalue}
is inherited from \PolyLib/.

\begin{verbatim}
void print_evalue(FILE *DST, const evalue *e, char **pname);
\end{verbatim}
The function \ai[\tt]{print\_evalue} dumps a human-readable
representation to the stream pointed to by \verb+DST+.
The argument \verb+pname+ points 
to an array of character strings representing the parameter names.
The array is assumed to be long enough.

\begin{verbatim}
int eequal(const evalue *e1, const evalue *e2);
\end{verbatim}
The function \ai[\tt]{eequal} return true (\verb+1+) if its
two arguments are structurally identical.
I.e., it does {\em not\/} check whether the two
(piecewise) quasi-polynomial represent the same function.

\begin{verbatim}
void reduce_evalue (evalue *e);
\end{verbatim}
The function \ai[\tt]{reduce\_evalue} performs some
simplifications on \ai[\tt]{evalue}s.
Here, we only describe the simplifications that are directly
related to the internal representation.
Some other simplifications are explained in
\citeN[Section~4.7.2]{Verdoolaege2005PhD}.
If the highest order coefficients of a \ai[\tt]{polynomial},
\ai[\tt]{fractional} or \ai[\tt]{flooring} are zero (possibly
after some other simplifications), then the size of the array
is reduced.  If only the constant term remains, i.e.,
the size is reduced to $1$ for  \ai[\tt]{polynomial} or to $2$
for the other types, then the whole node is replaced by
the constant term.
Additionally, if the argument of a \ai[\tt]{fractional}
has been reduced to a constant, then the whole node
is replaced by its partial evaluation.
A \ai[\tt]{relation} is similarly reduced if its second
branch or both its branches are zero.
Chambers with zero associated quasi-polynomials are
discarded from a \ai[\tt]{partition}.

\subsection{Generating Functions}

The representation of \rgf/s uses 
some basic types from the \ai[\tt]{NTL} library \shortcite{NTL}
for representing arbitrary precision integers
(\ai[\tt]{ZZ}) 
as well as vectors (\ai[\tt]{vec\_ZZ}) and matrices (\ai[\tt]{mat\_ZZ})
of such integers.
We further introduces a type \ai[\tt]{QQ} for representing a rational
number and use vectors (\ai[\tt]{vec\_QQ}) of such numbers.
\begin{verbatim}
struct QQ {
    ZZ	n;
    ZZ	d;
};

NTL_vector_decl(QQ,vec_QQ);
\end{verbatim}

Each term in a \rgf/ is represented by a \ai[\tt]{short\_rat}
structure.
\begin{verbatim}
struct short_rat {
    struct {
        /* rows: terms in numerator */
        vec_QQ  coeff;
        mat_ZZ  power;
    } n;
    struct {
        /* rows: factors in denominator */
        mat_ZZ  power;
    } d;
};
\end{verbatim}
The fields \ai[\tt]{n} and \ai[\tt]{d} represent the
numerator and the denominator respectively.
Note that in our implementation we combine terms
with the same denominator.
In the numerator, each element of \ai[\tt]{coeff} and each row of \ai[\tt]{power}
represents a single such term.
The vector \ai[\tt]{coeff} contains the rational coefficients
$\alpha_i$ of each term.
The columns of \ai[\tt]{power} correspond to the powers
of the variables.
In the denominator, each row of \ai[\tt]{power}
corresponds to the power $\vec b_{ij}$ of a 
factor in the denominator.

\begin{example}
Figure~\ref{fig:rat}
shows the internal representation of
$$
\frac{\frac 3 2 \, x_0^2 x_1^3 + 2 \, x_0^5 x_1^{-7}}
{ (1 - x_0 x_1^{-3}) (1 - x_1^2)}
.
$$

\begin{figure}
\begin{center}
\begin{minipage}{0cm}
\begin{xy}
*\hbox{
\tt
\begin{tabular}{|c|c|c|}
\hline
n.coeff & 3 & 2 \\
\cline{2-3}
	& 2 & 1 \\
\hline
n.power & 2 & 3 \\
\cline{2-3}
	& 5 & -7 \\
\hline
d.power & 1 & -3 \\
\cline{2-3}
	& 0 & 2 \\
\hline
\end{tabular}
}+UC*++!D\hbox{\tt short\_rat}
\end{xy}
\end{minipage}
\end{center}
\caption{Representation of
$
\left(\frac 3 2 \, x_0^2 x_1^3 + 2 \, x_0^5 x_1^{-7}\right)
/ \left( (1 - x_0 x_1^{-3}) (1 - x_1^2)\right)
$.}
\label{fig:rat}
\end{figure}

\end{example}

The whole \rgf/ is represented by a \ai[\tt]{gen\_fun}
structure.
\begin{verbatim}
typedef std::set<short_rat *,
                 short_rat_lex_smaller_denominator > short_rat_list;

struct gen_fun {
    short_rat_list term;
    Polyhedron *context;

    void add(const QQ& c, const vec_ZZ& num, const mat_ZZ& den);
    void add(short_rat *r);
    void add(const QQ& c, const gen_fun *gf,
                barvinok_options *options);
    void substitute(Matrix *CP);
    gen_fun *Hadamard_product(const gen_fun *gf,
                              barvinok_options *options);
    void print(std::ostream& os,
               unsigned int nparam, char **param_name) const;
    operator evalue *() const;
    ZZ coefficient(Value* params, barvinok_options *options) const;
    void coefficient(Value* params, Value* c) const;

    gen_fun(Polyhedron *C);
    gen_fun(Value c);
    gen_fun(const gen_fun *gf);
    ~gen_fun();
};
\end{verbatim}
A new \ai[\tt]{gen\_fun} can be constructed either as empty \rgf/ (possibly
with a given context \verb+C+), as a copy of an existing \rgf/ \verb+gf+, or as 
constant \rgf/ with value for the constant term specified by \verb+c+.
\\
The first \ai[\tt]{gen\_fun::add} method adds a new term to the \rgf/,
described by the coefficient \verb+c+, the numerator \verb+num+ and the
denominator \verb+den+.
It makes all powers in the denominator lexico-positive,
orders them in lexicographical order and inserts the new
term in \ai[\tt]{term} according to the lexicographical
order of the combined powers in the denominator.
The second \ai[\tt]{gen\_fun::add} method adds \verb+c+ times \verb+gf+
to the \rgf/.
\\
The method \ai[\tt]{gen\_fun::operator evalue *} performs
the conversion from \rgf/ to \psp/ explained in 
\citeN[Section~4.5.5]{Verdoolaege2005PhD}.
The \ai[\tt]{Polyhedron} \ai[\tt]{context} is the superset
of all points where the enumerator is non-zero used during this conversion,
i.e., it is the set $Q$ from \citeN[Equation~4.31]{Verdoolaege2005PhD}.
If  \ai[\tt]{context} is \verb+NULL+ the maximal
allowed context is assumed, i.e., the maximal
region with lexico-positive rays.  
\\
The method \ai[\tt]{gen\_fun::coefficient} computes the coefficient
of the term with power given by \verb+params+ and stores the result
in \verb+c+.
This method performs essentially the same computations as
\ai[\tt]{gen\_fun::operator evalue *}, except that it adds extra
equality constraints based on the specified values for the power.
\\
The method \ai[\tt]{gen\_fun::substitute} performs the
\ai{monomial substitution} specified by the homogeneous matrix \verb+CP+
that maps a set of ``\ai{compressed parameter}s'' \shortcite{Meister2004PhD}
to the original set of parameters.
That is, if we are given a \rgf/ $G(\vec z)$ that encodes the
explicit function $g(\vec i')$, where $\vec i'$ are the coordinates of
the transformed space, and \verb+CP+ represents the map
$\vec i = A \vec i' + \vec a$ back to the original space with coordinates $\vec i$,
then this method transforms the \rgf/ to $F(\vec x)$ encoding the
same explicit function $f(\vec i)$, i.e., 
$$f(\vec i) = f(A \vec i' + \vec a) = g(\vec i ').$$
This means that the coefficient of the term 
$\vec x^{\vec i} = \vec x^{A \vec i' + \vec a}$ in $F(\vec x)$ should be equal to the
coefficient of the term $\vec z^{\vec i'}$ in $G(\vec z)$.
In other words, if
$$
G(\vec z) =
\sum_i \epsilon_i \frac{\vec z^{\vec v_i}}{\prod_j (1-\vec z^{\vec b_{ij}})}
$$
then
$$
F(\vec x) =
\sum_i \epsilon_i \frac{\vec x^{A \vec v_i + \vec a}}
                       {\prod_j (1-\vec x^{A \vec b_{ij}})}
.
$$
\\
The method \ai[\tt]{gen\_fun::Hadamard\_product} computes the
\ai{Hadamard product} of the current \rgf/ with the \rgf/ \verb+gf+,
as explained in \citeN[Section~4.5.2]{Verdoolaege2005PhD}.

\subsection{Counting Functions}
\label{a:counting:functions}

Our library provides essentially three different counting functions:
one for non-parametric polytopes, one for parametric polytopes
and one for parametric sets with existential variables.
The old versions of these functions have a ``\ai[\tt]{MaxRays}''
argument, while the new versions have a more general
\ai[\tt]{barvinok\_options} argument.
For more information on \ai[\tt]{barvinok\_options}, see Section~\ref{a:options}.

\begin{verbatim}
void barvinok_count(Polyhedron *P, Value* result, 
                    unsigned NbMaxCons);
void barvinok_count_with_options(Polyhedron *P, Value* result,
                                 struct barvinok_options *options);
\end{verbatim}
The function \ai[\tt]{barvinok\_count} or
\ai[\tt]{barvinok\_count\_with\_options} enumerates the non-parametric
polytope \verb+P+ and returns the result in the \ai[\tt]{Value}
pointed to by \verb+result+, which needs to have been allocated
and initialized.
If \verb+P+ is a union, then only the first set in the union will
be taken into account.
For the meaning of the argument \verb+NbMaxCons+, see
the discussion on \ai[\tt]{MaxRays} in Section~\ref{a:options}.

The function \ai[\tt]{barvinok\_enumerate} for enumerating
parametric polytopes was meant to be
a drop-in replacement of \PolyLib/'s \ai[\tt]{Polyhedron\_Enumerate}
function.
Unfortunately, the latter has been changed to
accept an extra argument in recent versions of \PolyLib/ as shown below.
\begin{verbatim}
Enumeration* barvinok_enumerate(Polyhedron *P, Polyhedron* C, 
                                unsigned MaxRays);
extern Enumeration *Polyhedron_Enumerate(Polyhedron *P,
	    Polyhedron *C, unsigned MAXRAYS, char **pname);
\end{verbatim}
The argument \verb+MaxRays+ has the same meaning as the argument
\verb+NbMaxCons+ above.
The argument \verb+P+ refers to the $(d+n)$-dimensional
polyhedron defining the parametric polytope.
The argument \verb+C+ is an $n$-dimensional polyhedron containing
extra constraints on the parameter space.
Its primary use is to indicate how many of the dimensions
in \verb+P+ refer to parameters as any constraint in \verb+C+
could equally well have been added to \verb+P+ itself.
Note that the dimensions referring to the parameters should
appear {\em last}.
If either \verb+P+ or \verb+C+ is a union,
then only the first set in the union will be taken into account.
The result is a newly allocated \ai[\tt]{Enumeration}.
As an alternative we also provide a function 
(\ai[\tt]{barvinok\_enumerate\_ev} or
\ai[\tt]{barvinok\_enumerate\_with\_options}) that returns
an \ai[\tt]{evalue}.
\begin{verbatim}
evalue* barvinok_enumerate_ev(Polyhedron *P, Polyhedron* C, 
                              unsigned MaxRays);
evalue* barvinok_enumerate_with_options(Polyhedron *P,
        Polyhedron* C, struct barvinok_options *options);
\end{verbatim}

For enumerating parametric sets with existentially quantified variables,
we provide two functions:
\ai[\tt]{barvinok\_enumerate\_e},
and
\ai[\tt]{barvinok\_enumerate\_isl}.
\begin{verbatim}
evalue* barvinok_enumerate_e(Polyhedron *P,
          unsigned exist, unsigned nparam, unsigned MaxRays);
evalue* barvinok_enumerate_e_with_options(Polyhedron *P, 
          unsigned exist, unsigned nparam,
          struct barvinok_options *options);
evalue *barvinok_enumerate_isl(Polyhedron *P, 
          unsigned exist, unsigned nparam,
          struct barvinok_options *options);
evalue *barvinok_enumerate_scarf(Polyhedron *P,
          unsigned exist, unsigned nparam,
          struct barvinok_options *options);
\end{verbatim}
The first function tries the simplification rules from
\citeN[Section~4.6.2]{Verdoolaege2005PhD} before resorting to the method
based on \indac{PIP} from \citeN[Section~4.6.3]{Verdoolaege2005PhD}.
The second function immediately applies the technique from
\citeN[Section~4.6.3]{Verdoolaege2005PhD}.
The argument \verb+exist+ refers to the number of existential variables,
whereas
the argument \verb+nparam+ refers to the number of parameters.
The order of the dimensions in \verb+P+ is:
counted variables first, then existential variables and finally
the parameters.
The function \ai[\tt]{barvinok\_enumerate\_scarf} performs the same
computation as the function \ai[\tt]{barvinok\_enumerate\_scarf\_series}
below, but produces an explicit representation instead of a generating function.

\begin{verbatim}
gen_fun * barvinok_series(Polyhedron *P, Polyhedron* C, 
                          unsigned MaxRays);
gen_fun * barvinok_series_with_options(Polyhedron *P,
    Polyhedron* C, barvinok_options *options);
gen_fun *barvinok_enumerate_e_series(Polyhedron *P,
                  unsigned exist, unsigned nparam,
                  barvinok_options *options);
gen_fun *barvinok_enumerate_scarf_series(Polyhedron *P,
                          unsigned exist, unsigned nparam,
                          barvinok_options *options);
\end{verbatim}
The function 
\ai[\tt]{barvinok\_series} or
\ai[\tt]{barvinok\_series\_with\_options} enumerates parametric polytopes
in the form of a \rgf/.
The polyhedron \verb+P+ is assumed to have only
revlex-positive rays.
\\
The function \ai[\tt]{barvinok\_enumerate\_e\_series} computes a
generating function for the number of point in the parametric set
defined by \verb+P+ with \verb+exist+ existentially quantified
variables using the \ai{projection theorem}, as explained
in \autoref{s:projection}.
The function \ai[\tt]{barvinok\_enumerate\_scarf\_series} computes a
generating function for the number of point in the parametric set
defined by \verb+P+ with \verb+exist+ existentially quantified
variables, which is assumed to be 2.
This function implements the technique of
\shortciteN{Scarf2006Neighborhood} using the \ai{neighborhood complex}
description of \shortciteN{Scarf1981indivisibilities:II}.
It is currently restricted to problems with 3 or 4 constraints involving
the existentially quantified variables.

\subsection{Auxiliary Functions}

In this section we briefly mention some auxiliary functions
available in the \barvinok/ library.

\begin{verbatim}
void Polyhedron_Polarize(Polyhedron *P);
\end{verbatim}
The function \ai[\tt]{Polyhedron\_Polarize} 
polarizes its argument and is explained
in \citeN[Section~4.4.2]{Verdoolaege2005PhD}.

\begin{verbatim}
int unimodular_complete(Matrix *M, int row);
\end{verbatim}
The function \ai[\tt]{unimodular\_complete} extends
the first \verb+row+ rows of
\verb+M+ with an integral basis of the orthogonal complement
as explained in Section~\ref{s:completion}.
Returns non-zero
if the resulting matrix is unimodular\index{unimodular matrix}.

\begin{verbatim}
int DomainIncludes(Polyhedron *D1, Polyhedron *D2);
\end{verbatim}
The function \ai[\tt]{DomainIncludes} extends
the function \ai[\tt]{PolyhedronIncludes}
provided by \PolyLib/ 
to unions of polyhedra.
It checks whether every polyhedron in the union {\tt D2}
is included in some polyhedron of {\tt D1}.

\begin{verbatim}
Polyhedron *DomainConstraintSimplify(Polyhedron *P, 
                                     unsigned MaxRays);
\end{verbatim}
The value returned by
\ai[\tt]{DomainConstraintSimplify} is a pointer to
a newly allocated \ai[\tt]{Polyhedron} that contains the 
same integer points as its first argument but possibly
has simpler constraints.
Each constraint $ g \sp a x \ge c $
is replaced by $ \sp a x \ge \ceil{ \frac c g } $,
where $g$ is the \ac{gcd} of the coefficients in the original
constraint.
The \ai[\tt]{Polyhedron} pointed to by \verb+P+ is destroyed.

\begin{verbatim}
Polyhedron* Polyhedron_Project(Polyhedron *P, int dim);
\end{verbatim}
The function \ai[\tt]{Polyhedron\_Project} projects
\verb+P+ onto its last \verb+dim+ dimensions.

\begin{verbatim}
Matrix *left_inverse(Matrix *M, Matrix **Eq);
\end{verbatim}
The \ai[\tt]{left\_inverse} function computes the left inverse
of \verb+M+ as explained in Section~\ref{s:inverse}.

\sindex{reduced}{basis}
\sindex{generalized}{reduced basis}
\begin{verbatim}
Matrix *Polyhedron_Reduced_Basis(Polyhedron *P,
                                 struct barvinok_options *options);
\end{verbatim}
\ai[\tt]{Polyhedron\_Reduced\_Basis} computes
a \ai{generalized reduced basis} of {\tt P}, which
is assumed to be a polytope, using the algorithm
of~\shortciteN{Cook1993implementation}.
See \autoref{s:feasibility} for more information.
The basis vectors are stored in the rows of the matrix returned.

\begin{verbatim}
Vector *Polyhedron_Sample(Polyhedron *P,
                          struct barvinok_options *options);
\end{verbatim}
\ai[\tt]{Polyhedron\_Sample} returns an \ai{integer point} of {\tt P}
or {\tt NULL} if {\tt P} contains no integer points.
The integer point is found using the algorithm
of~\shortciteN{Cook1993implementation} and uses
\ai[\tt]{Polyhedron\_Reduced\_Basis} to compute the reduced bases.
See \autoref{s:feasibility} for more information.
