\section{Implementation details}

\subsection{The integer points in the fundamental parallelepiped of a simple cone}

This section is based on \shortciteN[Lemma 5.1]{Barvinok1992volume} and
\shortciteN{Koeppe2006experiments}.

\sindex{simple}{cone}
\sindex{open}{facet}
\sindex{open}{ray}
\sindex{explicit}{representation}
In this section we will deal exclusively with \ai{simple cone}s,
i.e. $d$-dimensional cones with $d$ extremal rays and $d$ facets.
\index{open facet}%
Some of the facets of these cones may be open.
Since we will mostly be dealing with cones in their
\ai{explicit representation}, we will have occasion to speak of
``\ai{open ray}s'', by which we will mean that the facet not
containing the ray is open.  (There is only one such facet because the cone
is simple.)

\sindex{fundamental}{parallelepiped}
\begin{definition}[Fundamental parallelepiped]
Let $K = \vec v + \poshull \lb\, \vec u_i \,\rb$ be 
a closed (shifted) cone, then the \defindex{fundamental parallelepiped} $\Pi$
of $K$ is
$$
\Pi = \vec v +
\lb\, \sum_i \alpha_i \vec u_i \mid 0 \leq \alpha_i < 1 \,\rb
.
$$
If some of the rays $\vec u_i$ of $K$ are open, then the constraints on
the corresponding coefficient $\alpha_i$ are such that $0 < \alpha_i \le 1$.
\end{definition}

\begin{lemma}[Integer points in the fundamental parallelepiped of a simple cone]
Let $K = \vec v + \poshull \lb\, \vec u_i \,\rb$ be a closed simple cone
and let $A$ be the matrix with the generators $\vec u_i$ of $K$
as rows.
Furthermore let $V A W^{-1} = S = \diag \vec s$ be the \indac{SNF} of $A$.
Then the integer points in the fundamental parallelepiped of $K$ are given
by
\begin{eqnarray}
\label{eq:parallelepiped}
\vec w^\T & = & \vec v^\T + \fractional{(\vec k^\T W - \vec v^\T) A^{-1}} A
\\
\nonumber
& = &
\vec v^T +
\sum_{i=1}^d
    \fractional{\sps{\sum_{j=1}^d k_j \vec w^T_j - \vec v^\T}{\vec u^*_i}} \vec u_i,
\end{eqnarray}
where $\vec u^*_i$ are the columns of $A^{-1}$ and $k_j \in \ZZ$ ranges
over $0 \le k_j < s_j$.
\end{lemma}

\begin{proof}
Since $0 \le \fractional{x} < 1$, it is clear that each such $\vec w$
lies inside the fundamental parallelepiped.
Furthermore,
\begin{eqnarray*}
\vec w^\T & = & \vec v^\T + \fractional{(\vec k^\T W - \vec v^\T) A^{-1}} A
\\
& = &
\vec v^T + 
\left(
(\vec k^\T W - \vec v^\T) A^{-1} - \floor{(\vec k^\T W - \vec v^\T) A^{-1}}
\right) A
\\
& = &
\underbrace{\vec k^\T W\mathstrut}_{\in \ZZ^{1\times d}}
+
\underbrace{\floor{(\vec k^\T W - \vec v^\T) A^{-1}}}_{\in \ZZ^{1\times d}}
\underbrace{A\mathstrut}_{\in \ZZ^{d\times d}} \in \ZZ^{1\times d}.
\end{eqnarray*}
Finally, if two such $\vec w$ are equal, i.e., $\vec w_1 = \vec w_2$,
then
\begin{eqnarray*}
\vec 0^\T = \vec w_1^\T - \vec w_2^\T
& = & \vec k_1^\T W - \vec k_2^\T W + \vec p^\T A
\\
& = & \left(\vec k_1^\T - \vec k_2^\T \right) W + \vec p^\T V^{-1} S W,
\end{eqnarray*}
with $\vec p \in \ZZ^d$,
or $\vec k_1 \equiv \vec k_2 \mod \vec s$, i.e., $\vec k_1 = \vec k_2$.
Since $\det S = \det A$, we obtain all points in the fundamental parallelepiped
by taking all $\vec k \in \ZZ^d$ satisfying $0 \le k_j < s_j$.
\end{proof}

If the cone $K$ is not closed then the coefficients of the open rays
should be in $(0,1]$ rather than in $[0,1)$.
In (\ref{eq:parallelepiped}),
we therefore need to replace the fractional part $\fractional{x} = x - \floor{x}$
by $\cractional{x} = x - \ceil{x-1}$ for the open rays.

\begin{figure}
\intercol=1.2cm
\begin{xy}
<\intercol,0pt>:<0pt,\intercol>::
\POS@i@={(0,-3),(0,0),(4,2),(4,-3)},{0*[grey]\xypolyline{*}}
\POS@i@={(0,-3),(0,0),(4,2)},{0*[|(2)]\xypolyline{}}
\POS(-1,0)\ar(4.5,0)
\POS(0,-3)\ar(0,3)
\POS(0,0)\ar@[|(3)](0,-1)
\POS(0,0)\ar@[|(3)](2,1)
\POS(0,-1)\ar@{--}@[|(2)](2,0)
\POS(2,1)\ar@{--}@[|(2)](2,0)
\POS(0,0)*{\bullet}
\POS(1,0)*{\bullet}
\end{xy}
\caption{The integer points in the fundamental parallelepiped of $K$}
\label{f:parallelepiped}
\end{figure}

\begin{example}
Let $K$ be the cone
$$
K = \sm{0 \\ 0} + \poshull \lb\, \sm{2 \\ 1}, \sm{0 \\ -1} \,\rb
,
$$
shown in Figure~\ref{f:parallelepiped}.
Then
$$
A = \sm{2 & 1\\0 & -1} \qquad A^{-1} = \sm{1/2 & 1/2 \\ 0 & -1 }
$$
and
$$
\sm{1 & 0 \\ 1 & 1 } \sm{2 & 1\\0 & -1} = \sm{1 & 0 \\ 0 & 2} \sm{2 & 1 \\ 1 & 0}.
$$
We have $\det A = \det S = 2$ and
$\vec k_1^\T = \sm{0 & 0}$ and $\vec k_2^\T = \sm{0 & 1}$.
Therefore,
$$
\vec w_1^\T = \fractional{\sm{0 & 0} \sm{2 & 1 \\ 1 & 0} \sm{1/2 & 1/2 \\ 0 & -1 }}
\sm{2 & 1\\0 & -1} = \sm{0 & 0}
$$
and
\begin{eqnarray*}
\vec w_2^\T & = & 
\fractional{\sm{0 & 1} \sm{2 & 1 \\ 1 & 0} \sm{1/2 & 1/2 \\ 0 & -1 }}
\sm{2 & 1\\0 & -1}
\\
& = &
\sm{1/2 & 1/2} \sm{2 & 1\\0 & -1} = \sm{1 & 0}.
\end{eqnarray*}
\end{example}
