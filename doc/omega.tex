\section{\texorpdfstring{\protect\Omegalib/ interface}
{Omega interface}}

The \ai[\tt]{barvinok} distribution includes an interface
to \Omegalib/ \shortcite{Omega_lib} \ai[\tt]{occ}, an extension
of \ai[\tt]{oc} \shortcite{Omega_calc}.
The extension adds the operations shown in Figure~\ref{f:unary}.
Here are some examples:
\begin{verbatim}
symbolic n, m; 
P := { [i,j] : 0 <= i <= n and i <= j <= m };
card P;

P := {[i,j] : 0 <= i < 4*n-1 and 0 <= j < n and
              n-1 <= i+j <= 3*n-2 };
C1 := {[i,j] : 0 <= i < 4*n-1 and 0 <= j < n and
               2*n-1 <= i+j <= 4*n-2 and i <= 2*n-1 };

count_lexsmaller P within C1;

vertices C1;

bmax { [i] -> 2*n*i - n*n + 3*n - 1/2*i*i - 3/2*i-1 :
        (exists j : 0 <= i < 4*n-1 and 0 <= j < n and
                    2*n-1 <= i+j <= 4*n-2 and i <= 2*n-1 ) };

sum { [i,j] -> i*j + n*i*i*j : i,j >= 0 and 5i + 27j <= n+m };
\end{verbatim}

\begin{figure}
\begin{tabular}{lp{0.25\textwidth}p{0.5\textwidth}}
Name & Syntax & Explanation
\\
\hline
Card & \ai[\tt]{card} $r$ & Computes the number of integer points in $r$ and
prints the result to standard output
\\
Card & \ai[\tt]{card} $r$ {\tt using} \ai[\tt]{parker} &
Computes the number of integer points in $r$ and
prints the result to standard output
using the method of \shortciteN{Parker2004}
\\
Ranking & \ai[\tt]{ranking} $r$ & Computes the rank function of $r$ and
prints the result to standard output
\shortcite{Loechner2002,Turjan2002}
\\
Predecessors & \ai[\tt]{count\_lexsmaller} $r$ \ai[\tt]{within} $d$ &
Computes a function from the elements of $d$ to
the number of elements of $r$ that are lexicographically
smaller than that element and
prints the result to standard output.
\\
Vertices & \ai[\tt]{vertices} $r$ &
Computes the parametric vertices of $r$ using \PolyLib/ \shortcite{Loechner1999}.
\\
Bernstein & \ai[\tt]{bmax} $f$ &
Computes the \ai{Bernstein coefficient}s of the function $f$ over its domain
and removes the redundant coefficients by calling
\ai[\tt]{piecewise\_lst::maximize}.  The results are printed to standard
output.  See the example for how to specify the function $f$.
\\
Sum & \ai[\tt]{sum} $f$ &
Computes the sum of the given polynomial $f$ over its domain
using \ai[\tt]{barvinok\_summate}.
\end{tabular}
\caption{Extra relational operations of {\tt occ}}
\label{f:unary}
\end{figure}
